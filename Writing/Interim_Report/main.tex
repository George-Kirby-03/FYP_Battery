\documentclass[11pt]{article}
\usepackage[utf8]{inputenc}
\usepackage{graphicx}
\usepackage[margin=1in]{geometry}
\usepackage{amsmath}
\usepackage{hyperref}
\usepackage{chemformula}
\usepackage{amssymb}
\setlength{\parskip}{0.8em plus 0.2em minus 0.1em}
\setlength{\parindent}{0pt}

\title{
    
    Understanding the problem\\[1cm]
    \includegraphics[width=0.25\textwidth]{../Images/UOS.png}\\[2cm]
    \textbf{Implamenting multistage constant-current charging methods in lithium based batterys to reduce degredation} \\[1em]
}
\author{George W. Kirby \\[2em] \textit{200328186} \\[2em] }
\date{\today}
\linespread{1.5}
\begin{document}
\maketitle
\vfill 
\begin{center}
    \textbf{Supervisor:} Dr. Ross Drummond 
\end{center}
\newpage
{
  \setlength{\parskip}{0pt}
  \tableofcontents
  \newpage
}

\section{Introduction}
There are curently over 7 billion people with acess to electricity \cite{chen_physics-informed_2025}, the global consuption of energy is rising by $4.5\times10^{16}$ Joules every year; 77\% of which is from
utilising non-renewable sources. With the affects of non-renwables on the environment still not fully realised, coupled with the concern of their finitie-life nature, poses a constant need to increase usage of renewable sources.
However, a large set of renwable generation methods fall under the category of veriable renewable energy (VRE) sources \cite{guo_effects_2024}, thus requring a robust form of energy storage to solve the itermitten availability -
methods of storage do exist and suitable solutions depend on the specifics and locations; as battery techonolgy increases, the use of battery as a storage medium for the energy sector are ever more increasing.  
There is also the rapid increase in elecrict vehicle production, with a push to increase the adoption of EV's, 17.3 million cars were produced in 2024 \cite{li_evolution_2024} alone with a recent 20\% average year-on
increase. Lithium Ion batterys (LIB's) are the most common currently used batterys due to their desirbale characyeristics in energy density, aging properties, cost and more.

EV's and the energy sector account now account for 90\% of the total lithium ion battery demand since 2016, the total lithium-ion battery demand is 10-times larger \cite{noauthor_executive_nodate} since 2016. Different resources predict different forcasts of total battery 
growth \cite{chinnam_fast-charging_2021,lucu_data-driven_2020}, regardless, it appears the rate of demand wil not be decreasing in the near future. 

LIB's, like all batterys and decrease in performance and capacity \cite{palacin_why_2016} until they are deemed unsuitable for their current use. This means eventually, all current batterys in use today will need to be disposed of, given that
over 660,000 tonnes of earth metals were mined in 2023 alone, a strong case for slowing down the need for mining new materials is made, whilst keeping up with the demand (cite the iea thing). It's noted that battery chemistry is still a developing feild, thus newer batteries may use less scarce materials, thus hard to
truly predict the sustaniability of mining these materials, for example, some reports (cite faraday) say cobalt supplies could be used entirley by 2040. Solutions do exits, recycling methods are been developed in order to extract the materials, termed
battery metal recycling (BMR), however there are wildly different sources suggesting how much of current batterys actually are recyled (one showing 5\%), (other showing 90\%), and the techonolgy to do so is still a devloping feild, there are many different methods (cite issurs), since most of the rare earth
elements are within the batterys cathode, methods such as hydrometallurgical process can extract these, but at the cost of severe reduction in lithum yeild (cite enji) - although there are promising methods with some proposing a lithium recovery needing less than 40\% the energy compared to mining the virgin material 
(cite enji yoo) whilst still recovering nickle and cobalt. There is also the rising interest in reusing the batterys before recyling, these batterys are ferrerd to as second-life batteries (SLB's); in the EV use-case, the majority of the batteries are deemed finished once their usable capacity drops below 80\% (cite),
but often, due to the other parameters of the batteries needing to be of high standards for EV use, the 'dead' batteries can still provide a very usefull serivce in less demanding applications, a common one being energy storage systems (ESS) \cite{mattia_lithium-ion_2025}, with claims resuing can reduce CO2 emmisions by
56\% as apposed to using natural fuel gas in situe. However there are again challanges associated, for saftey there is the need for robust testing, asorting the batteries based on their current health states, non automised methods of extracting cells from their original pack. These additional steps mean the price of SLB's 
could become at a point where they are not much cheaper than new batteries, making the solution less attractive to many sectors \cite{mattia_lithium-ion_2025}.

In summary, there are methods that are devloping to reduce the need for mining new materials and decreasing recyling emrgy, but ultimately, providing methods to increase the usable life of the current batteries serves as a direct way reduce the impacts of batteries themselves and to meet their growing demand.
If just 1 extra life cycle could be added to all EV's currently on the road (~60 million (citr)), this could power all of the UK's energy needs for one day (cite govengy); as most batterys cycle life are well over 1000, any incament in cycle life can be greatly beneficial.

Much research has been done into both modeling LIB's and optimising their usage from both an energy and aging perspective. 


Moddeling and prediciting how batteries degreade over time is pivitol for the growing use of lithium batteries as it allows informed descisions on how to best charge and use battereis to prolong life, as well allow optimisations on the charging throughout the batteries life.

\subsection{Project Aims}
This project in essance aims first to characterise any battereies behaviour by perameterisation without ex-situe tests. This data will then be used in combination with work previously done to form the charging protocol based off minimising different objectve functions (the level of affects the previously proposed objective functions will be also verified by acting these tests). Data from this will be used to form the basis of a machine learning model in the hopes this model can, over time, allow for a much simpler method of predicting a batteries future behaviour.
As will be discussed in the sections below, characterising batteries behaviour can require upto nearly 30 parameters about the specific battery in order to see the affects of degredation; even then, due to minor errors in soltuons can fail to predict future properties.

Figure x highlits the original idealised goal, to have a purley data driven model that can predict the output voltage over any cycle, thus been able to derrive an idealised charging current to maxamise the lifespan. 
\begin{figure}[ht]
    \centering
    \includegraphics[width=0.5\textwidth]{../Images/nn_battery.png}
    \caption{Inital end objective: Black Box Battery}
\end{figure}

\subsubsection{Objectives}
\begin{itemize}
    \item Analyse ICLOCS2 and other methods in parameterising batteries during realtime use
    \item Modify previously developed charging methods for running in a lab to rest actual results, inspect the affects of each method
    \item See how affective adapting the charging profile over the degredadtion is in increasing battery life
    \item Propose and test new charging based on results of prevouls developed profiles
    \item Feed results to a data driven model to find a new optimal charging profile
\end{itemize}

\section{Literature Review}
\subsection{Lithium based battery degredation modes}
\subsubsection*{Basic Background of Lithium Ion Batteries}

During my time in trying to understand how batteries work and enginering appraches to model,

Lithium-ion based batteryies are favoruable largley due to the highly reducitve nature of Lithium, taking the potential relative to hydorgens reduction, lithiums half-reaction is $-3.01V$ (cite gregg) relative to hydrogens half-reaction, meaning lithium can release its outershell electron relativley easy. Any cells potential difference is the diference in potentals of the negative and posoitve solid state electrodes $V_t = \phi_s^+ (t) -  \phi_s^- (t) - IR_{cc}$ where $R_{cc}$ is the current collector or tabs ohmic resistance (cite liuying), so having negative half reduction of lithium, allows a varierty of postive electordes (where the oxidation occurs), to yeild a high battery voltage. Coupled with Lithiums low molecular weight, a high energy density cell can be achived to. The force produced by the potential difference between the electrodes drives the elctrons via the external circuit, providing useful work, allowing lithium batteries to act as an energy storage medium.


Lithium batteries all follow the the same basic pricipble of operation, as show in .... They consist of a posotive and negative elctrode, eleoctrolite, a seperator and current collectors. Unlike many other battery chemistries, where the metal ions undergo chemical reactions at the electrodes which change the composition of the electrodes, lithium ions intercolate and deintercalate within the electrodes - they are in essance absorbed to and from the electrode structures (cite greg p). The negative electrode is usually of a grpahine based structure, allowing up to 1 lithium atom to be stored per 6 carbon atoms, whilst the posotive electrode can vary greatly between different lithium batteries; the most common posotive electrodes are often composed of transitional metal oxides such as LCO, LMO, NMC, NCA (cite Alex K Hoech et al). The lithium state in the graphine is similar to that of idividual lithium atoms, thus the elctrode potention of the negative electoride is very similar to that of pure lithium (~ -3V) (cite liquin), although its lithium storage abilities are usually lower than the porsotive electrodes. The lithium ions within the posotive electrodes are in a much lower energy state, thus their reaction potential ranges from -1 -- 2V, therefore, overall there is a large potential difference created by the want of the posotive electrode gaining electrons and the lithium ions, and the want of the lithium within the carbon of the negative electrode to give up the lithium ions and electrons (relative to each other). In general, the negative electrodes used currently are carbon based, there capcity to hold the lithium (per unit volume) is greater that the posotive electrodes used today as shown in figurexx (cite Alex), but as discussed later in this section the negative electrode is often the key limiting factor in degredation. The posotive electrode materials vary greatly in material, but since this electrode has the biggest electrode potential, it has a largest impact on the overall cell voltage, and thus energy density of the cell, in addition since per volume it holds less lithium, electrode material with higer lithium capcity will be of greater affect that improving the negative electrodes capcity. The two reactions with lithium with the battery electrodes chosen for this project are shown below (during dicharge, the reactions proceede from left to right, for the charging process, it's right to left):
\begin{align}
  \ch{LiFePO4 &<=> Li+ + e- + FePO4} \label{eq:lithium} \\
  \ch{C6 + Li+ + e- &<=> LiC6} \label{eq:sodium}
\end{align}


The electrolite is a medium in which the lithium ions can be transported via diffison to the electrodes, the electolite does not however allow electrons to flow, only via an extranl circuit can they flow, . The seperator acts as a structiral barrier to prevent the opoosite electrodes from toutching, which would allow both electrons and ions to flow, causing major fire risks from the uncotnrolled reactions. During the discharging process, a path for electrons is provided externally across the current collectors, (these collecors are not part of the reactions and are just a means to electrically connect the eleoctrolites to a usefull surface for soldering) the lithium stored within the negative electrode deintercalates, released as the posotive lithium ion, whilst the electron leaves the electrode, traveling acros the external circuit to the posotive electrode, whereby the lithium ions meet the electrons. During charging, the applied voltage at the terminals will be  above the difference of the electrode potentials, this applied overpotential causes the lithium ions within the posotive electrode to deintercalate, traveling across the electrolite to the negative electrode, where they intercalate within the graphine structure as the ions combine with electrons. The speed at which the ions flow is far slower than the electron transfer (cite greg), which contributes to the behaviour of lithium batteries.

Mention diffusion differences between charge and discharge (cite micheal j lain) 
Whiney well 
\subsubsection{Degredation methods}
Simon et al. \cite{jokane_lithium-ion_2022} simon e j o summerises the key 5 modes of degredation caused within lithium ion batteries

\textbf{Lithium plating} - During discharge, the lithium ions intercolate into the posotive electrode. Under certain charging conditions, the lithium ions within the electrolyte will join with the electrons outside the negative electrode, producing pure lithium metal - commonly reffered to as dentrites, depicted in figure ... There is still ongoing research into finding the key causes and which factors speed up this process, but it's most understood cause is when the negative electrode (carbon in the projects battery case) potential falls below that of lithiums own potnetial (\cite{yang_minimum_2021,lain_understanding_2021} shichun yang + Michael.J. Lain). From the Doyle-Fuller-Newman brefily described later (cite dfn), the electrode pot As previously mentioned, the lithinated carbon potnetintal is slighlty higher than that of pure lithium. From electrochemical models, the potential at the elctrodes is the sum of their open circuit potentials and their overpotential, overpotentials are caused from the dynamics of the battery, i.e during charge and discharge, thus, the overpotentials can drive the battery below that of lithium, aswell as the SoC of the battery which increases the natural potentials (cite zhiqiang). Meaning the more favourable reaction is to form lithium as apoosed to intercalate. The overpotential is largley due to the kenetic overpotential at the interface of the electrolyte, sei and electrodel; it's shown that low temperature, high currents and high SoC can cause the negative potential to drive lower (cite zhiqiang). Modelling aims to characterise and capture potentials to help with controling dendrite formation, methods from numerical electrochemical models like the p2D thing, using the butler-volmer kenetics to show key overpotentials, as well as state estimation approches for realtime - however they all require many parameters about the specific battereis. 

\textbf{SEI Layer growth} - A layer known as solid electyroytle interphase is formed as soon as the electrolyte soulution comes into contact with the negative electrode causing salts like \ch{Li_{2}CO_{3}} to produce acids, followerd by futher reactions; this barrier acts to preven electrons further reducing and using up the more of the electrolyte, whilst allowing passage of the lithium ions to intercolate\cite{bouguern_critical_2024}. If this SEI breaks appart, new SEI will form, taking more lithium up causing a reduction of lithum avaolable for charging. Having the battery at very high and very low SoC can cause the SEI layer to thicken also \cite{agubra_analysis_2014}. Sei formation takes up electrons within its reaction, and thus c

\textbf{Particle fracture} - The physical volume of the elctrodes can change during the interoclation and deintercalation can cause

\begin{figure}[ht]
    \centering
    \includegraphics[width=0.5\textwidth]{../Images/Lithium_Plating.png}
    \caption{Lithium deposits shown in lighter grey on the graphite electrode \cite{attia_closed-loop_2020}}
\end{figure}


Causes increased ageing and seftey risks, its the deposition of metallic lithoum on the anote surface, happens at high charging currents and low temperature. Since during charging, the lithium ions move , through the sei into the anode, if the ions cannot intercalate fast enough, they deposit and can become metallic lithium. Especially ehrn chargis is forces, local overpotential can causes the lithium plating, can cause dentrites


 \begin{itemize}
        \item SEI layer growth via pores $ \approx $ not really solvable, grows square root over time and cycle number
        \item Lithium plating \\
        Causes increased ageing and seftey risks, its the deposition of metallic lithoum on the anote surface, happens at high charging currents and low temperature.
        Since during charging, the lithium ions move , through the sei into the anode, if the ions cannot intercalate fast enough, they deposit and can become metallic lithium. 
        Especially ehrn chargis is forces, local overpotential can causes the lithium plating, can cause dentrites
        \textbf{this is one of the main constraints for the chargings profile}
        \item Active material loss (from parts mentiones above)
        \item SEI Brakeages \\
        Charging too \textbf{high} of a temperatures causes mechanical stress on the sei layer, causing it to crack and reform, consuming more lithium ions in the process. 
        Loose sei material can also float in the electrolyte, causing further issues.
        \item Electrolyte decomposition
    \end{itemize}
\subsection{Equivalent Circuit Model}

There are many ways a lithium battery is moddeled



This S2352152X2400584X nicley shows the overpotential of the battery relating to the differend of the output of the battery to the equlibrium potnetial of the battery. Shown as the summation of eletrolyte opverpotential, Li concentration overportnetial, kenetic overpotential.Attempts to estimate the anode voltage based off reduced order elkectrochemical models and by use of a state estimator. In the electrocemical models, the anode potential is equal to the kenetic overpotential, anode equilibrium potential and the electolite potebtial, (cite) shows the solid-phase electrode potnetial is equal to $U_{eq} + \mu _{ct} + \Phi _{electroylite}$ th(cite). 

and resistive overpotential. This are equated witin the ecm model as the polorixing rc, and r. 
0378775313020880 bib5 gives the 

The

Whilst the voltage dynamics of lithium batteries do depend on temperature, this project keeps the ambient temperature to 30deg nd moderatley low temperatures, thus realtime affects are minimal. However moddeling of the temperature is vital since temperature does directly affect degredadion and will help the ICLOCS parameterisation. cite the thermal paper.  The irreversible is much easier to directly relate to the ecm, by basic ohms law. However irreverisitbility can not be directly infered from the ecm model. Cite dodgey paper, equated it to proportional to $\partial U / \partial T$ where $U$ is the OCV and $T$ is the temperature, this value does change over the SoC of the battery and paper et all calculates the curve for a battery of similar to the test battery shown in figure x.x

\begin{equation}
    mCp\frac{d T}{dt} = \underbrace{I^{2}R_{0} + IV_{1}}_{irreversable} - \underbrace{IT\frac{\partial U}{\partial T}}_{reversable} - \underbrace{hAT}_{dissipation}
\end{equation}

cite greg, shows methods of extracting the element values for the ecm, but the ocv curve can be particualrily challenging, Especially given it hysterisis affects when charging and dischargign.   This is where iclocs and other modlers can come into play to help get these values 

cite Alexander Farman, shows that the OCV of lithium batteries can change throughout degredaditon, for phosphate batteries they showed a 20mV difference in regions of the SoC after aging of 500 cycles at 1C of charge and discharge. Coupled with the fact

\subsection{Charging methods}
There was a paper copmparing them in general, taljk aboutr the advantage of the different ones ect. 

Talk about CC, MultiStage CC, ConstantPower, George Tuckers minimisation functions

There exists many techniques to charge batteries with a graphical summary of the common methods shwon in figure xx, with the most common method known as CC-CV. This is where the bulk charge is done such that the applied overportnetial to the battery causes a constant current to flow, the voltage potential of the battery rises to a defined maximum point, the charging device then switches to a contant voltage charging mode, this fixed voltage is held until the charge current decays to a negligable or fixed amount - current still flows during this stage as there is still an overpotnetial between the batteries equlibrium state and the applied constant voltage. This is the simplest to implament without the need for an accurate SoC reading to risk overcharging (like would be with pure CC charging), CV charging can be too dangerous given the large current draw at low SoC. Some of the data used within this project is charged via CC-CV and is highlited in figure xx.
\begin{figure}[ht]
    \centering
    \includegraphics[width=0.5\textwidth]{../../Matlab/Figures/CCCVProtocol.png}
    \caption{Charging segemnt Lion}
\end{figure}

multistage

AC

\begin{figure}[ht]
    \centering
    \includegraphics[width=0.5\textwidth]{../Images/common_charge_graph.jpg}
    \caption{Common charge techniques, cite qian lin towards smarter}
\end{figure}

\subsection{Modelling degredation}
Talk about how stevesons linear model uses the change in voltage overtime (first 100 cycles). \textbf{Downside - } This requres a slow cycle to get the right features for the model, also only predicts EOL figure, useful for strong indication of battery prediction. However can not fully caupture the input affects of current, only fitting from the affects of it. Additionally does not show how specific characyeristics change.

Attia takes the use of steverson to great use, using the model to predic how well differenct charge cycles last 

There are three main categorical methods to predict the degredation of lithium ion battereies: Purley data driven methods, physics-based models and a hyrbdid (phuysics informed). Fujin Wang et al highlitghts the advantages of the hybrid approach, stating the main issue with pure data-driven and pure physics model, they also present a means of categorising the common hybrid approaches, given in figure x. 


Papers focusing on the physics based moedeling usually focus on a subset of degredadion modes, such as SEI thickness growth cite Mixed Mode Growth Model for the Solid Electrolyte Interface. Most models start with the Doyle-Fuller-Newman model cite Dole Fuller to describe the transport of lithium within the battery (including potentials), then further sub-modules representing degredadition mechanisms are intergrated to yeild the required models, the Dole Duller provides a set of coupled PDE's which when discritised with methods such as finite volume (cite pybam), serves as an approximation of the highly complex dynamics of the battery, others unclude the simpler single particle model and the single particle model with electrolyte SPMe. For example simon e j okane adds sei, li paritcle crackiong anmd lam as the the degredadion models, this can be represented similar to Jishnu Ayyana by representing the electrochamiecal dynamics as $\frac{d\mathbf{x}}{dt} = f(\mathbf{x, z ,u}, t; \mathbf{\alpha})$ and couplining to degradation dynamics $\frac{d\mathbf{z}}{dt} = f(\mathbf{z, x ,u}, t; \mathbf{\theta})$ where $\mathbf{x} \in \mathbb{R}^{n_1}$ and $\mathbf{z} \in \mathbb{R}^{n_1}$ (once discritised), representing the battery electrochemical and degredadtion state variables respecitlvey. These approaches can model intericate parts within the battery, but require a lot of parameters about each battery under test, additionally, the discritised solutions are only ever as accurate as the models used.  E j okane shows a more complete set of degredadtion modeling whulst Jishun applies ML to effectivlye learn the missing modes with good results.

Purley data-driven appraches have the ability to provide the most accurate predictions, however on their own suffer the problem that there not currently a large enough set of data to train on - this is a problem as lithium batteries have mutliple degredadtion modes as discuessed, as well as a highly dynamical system when viewed as 


Steverson aims to

\begin{itemize}
    \item Fmincon
    \item Greyest
    \item ICLOCS
\item \subsubsection{Machine Learning}
    \item General Understanding
    \item LSTM
    \item 
\end{itemize}

\section{Self Review}
Choosing this final year project has, and will continue to, be a big personal undertaking. I knew nothing about batteries other than the very basics when starting. Additionally I knew nothing about most machine learning, let alone modern nerual networking. Research has also been a struggle for me as it is easy to follow refferernces of refferences, as well as the disheartnment when some research is shwon that is either behind a paywall, or simply beyond my understanding. 

\begin{figure}[ht]
    \centering
    \includegraphics[width=1\textwidth]{../Images/Online Gantt 20251129.png}
    \caption{Inital end objective: Black Box Battery}
\end{figure}


\newpage
\nocite{*}
\bibliographystyle{../IEEEtran/IEEEtran}
\bibliography{My_Library}

\end{document}