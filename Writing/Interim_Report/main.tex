\documentclass[11pt]{article}
\usepackage[utf8]{inputenc}
\usepackage{graphicx}
\usepackage[margin=1in]{geometry}
\usepackage{amsmath}
\usepackage{hyperref}
\usepackage{chemformula}
\usepackage{amssymb}
\usepackage{subfig}
\setlength{\parskip}{0.8em plus 0.2em minus 0.1em}
\setlength{\parindent}{0pt}

\title{
    
    Understanding the problem\\[1cm]
    \includegraphics[width=0.25\textwidth]{../Images/UOS.png}\\[2cm]
    \textbf{Finding Machine learning methods to model the degredadition of lithium based batteries under multi-stage CC charging } \\[1em]
}
\author{George W. Kirby \\[2em] \textit{200328186} \\[2em] }
\date{\today}
\linespread{1.5}
\begin{document}
\maketitle
\vfill 
\begin{center}
    \textbf{Supervisor:} Dr. Ross Drummond 
\end{center}
\newpage
{
  \setlength{\parskip}{0pt}
  \tableofcontents
  \newpage
}

\section{Introduction}

The global consumption of energy is rising by $4.5\times10^{16}$ Joules every year; 77\% of which is from non-renewable sources \cite{noauthor_renewable_nodate}. With the effects of non-renewables on the environment still not fully realised, coupled with concerns over their finite nature, there is a constant need to increase the usage of renewable sources. However, a large set of renewable generation methods fall under the category of variable renewable energy (VRE) sources \cite{guo_effects_2024}, thus requiring a robust form of energy storage to solve their intermittent availability. As battery technology improves, the use of batteries as a storage medium for the energy sector is becoming increasingly prominent. There is also a rapid increase in electric vehicle production, with a push to increase the adoption of EVs. In 2024, 17.3 million EVs were produced \cite{li_evolution_2024}, alongside a recent average year-on-year increase of 20\%. Lithium-ion batteries (LIBs) are currently the most widely used batteries due to their desirable characteristics in energy density, ageing behaviour, cost, and more.

EVs and the energy sector now account for 90\% of total lithium-ion battery demand, and the total lithium-ion battery demand has increased ten-fold since 2016 \cite{noauthor_executive_nodate}. Different sources predict different forecasts for battery growth \cite{chinnam_fast-charging_2021,lucu_data-driven_2020}; regardless, the rate of demand does not appear to be decreasing in the near future.

LIBs decrease in performance and capacity over time \cite{palacin_why_2016}, until they are deemed unsuitable for their current use. Eventually, all batteries in use today will require disposal. In 2021, 436,000 tonnes of lithium were mined for batteries alone \cite{noauthor_battery_2023}, further highlighting the need to reduce reliance on mining while meeting demand. Battery chemistry is still a developing field, and newer batteries may use fewer scarce materials, making long-term sustainability difficult to predict. For example, some reports suggest cobalt reserves may be exhausted by 2040 \cite{gifford_lithium_nodate}. Recycling methods are being developed to recover battery materials, termed battery metal recycling (BMR) - however, sources vary widely, with estimates of current recycling rates ranging from as low as 5\%. Recycling technology is still developing, and many different process routes exist \cite{das_lithium-ion_2026}. Since most valuable metals are located in the cathode, hydrometallurgical processes can recover them, although often with reduced lithium yield \cite{hasan_advancing_2025}. Promising work has shown lithium recovery requiring less than 40\% of the energy needed to mine virgin material \cite{yoo_life-cycle_2023}, while still recovering nickel and cobalt.

There is also increasing interest in reusing batteries before recycling. These are referred to as second-life batteries (SLBs). In EV applications, batteries are typically retired when their usable capacity falls below 80\% \cite{wang_overview_2017}, yet they can still perform effectively in less demanding applications such as energy storage systems (ESS) \cite{mattia_lithium-ion_2025}. Re-use has been reported to reduce $CO_{2}$ emissions by up to 56\% compared with natural gas systems. However, challenges exist, including safety validation, cell sorting based on health, and the lack of automated pack disassembly methods. These additional steps may lead to SLBs being insufficiently cheaper than new batteries, limiting commercial appeal \cite{mattia_lithium-ion_2025}.

In summary, several methods are developing to reduce reliance on newly mined materials and lower recycling energy requirements, but extending the usable lifetime of existing batteries remains a direct and impactful way to reduce environmental burden while meeting rising demand. If just one additional life cycle could be added to all EVs currently on the road (~60 million \cite{noauthor_trends_nodate-2}), the stored energy would be enough to power the UK for one day \cite{noauthor_energy_nodate-1}. With most batteries capable of over 1000 cycles, any increase in cycle life can lead to significant benefit.

Modelling and predicting lithium-ion battery degradation is therefore pivotal, as it enables informed decisions on how best to charge and operate batteries to prolong life, as well as optimise usage across their lifetime.

\subsection{Project Aims}
This project aims first to characterise Lithium battereies behaviour by perameterisation, without ex-situe tests. This data will then be used in combination with work previously done to form the charging protocol based off minimising different objectve functions. Data from this will be used to form the basis of a machine learning model in the hopes this model can, over time, allow for a much simpler method of predicting a batteries future behaviour. If this model can show how different charging currents can affect degredation, a charging protocol which is derivived from minimsing such model will found and be tried in the labs to compare. If this is not achived, atleast this project can validate existing predicted charging to minimise degredaditon will be tested, and the parameterisation of the batteries during degredaditon will be researched by using optimisation tools.

Figure \ref{fig:bb} highlits the original idealised goal, to have a purley data driven model that can predict the output voltage over any cycle, thus been able to derrive an idealised charging current to maxamise the lifespan.

\begin{figure}[ht]
    \centering
    \includegraphics[width=0.5\textwidth]{../Images/nn_battery.png}
    \caption{Inital end objective: Black Box Battery modeling degredaditon}
    \label{fig:bb}
\end{figure}

\subsubsection{Objectives}
\begin{itemize}
    \item Analyse ICLOCS2 and other methods in parameterising batteries under realtime use
    \item Modify previously developed charging methods for running in a lab to rest actual results, inspect the affects of each method
    \item See how affective adapting the charging profile over the degredadtion is in increasing battery life
    \item Investigate ML methods to model the attained data to predict degredaditon
    \item Use the model to find a new charging method and validate
\end{itemize}

\section{Literature Review}
\subsection{Lithium based battery backgrounds}


\textbf{Notice:} Throughout the research and not coming from an electrochemical background, various literature denotes the meaning of \textit{potential} and \textit{overpotential} as slighly different given the context - there is likley equivalances, but of which the author of this report cant explain. For example cite(potential for confusion) shows a clear distinction between the common $\phi$ symbol for potential as the \textit{electrical potential}, however when talking about volts measured in the real world, the \textit{potential difference} is usually the \textit{electrochemical potential difference} (specifically of electrons), for two given points denoted by $ \Delta \tilde{\mu} = \Delta\mu + \Delta zF\phi$. Overpotential in some instaces, such as the Butler-Volmer model for current exhange, is a function of overpotential $\eta$ and from G. Plett\cite{noauthor_battery_nodate} is a difference in the current \textit{electrical potential} and equlibrium potential, yet papers like \cite{appiah_unravelling_2024} show overpotential more generally as the difference between any equlibium potential and current potential. Thus confirmation should be made in regards to what type of potential and overpotential. Assumptions are given that potentials are a form of energy state, whith a difference between two points causing a force, and overpotneital as additional energy for an event to occur.


Lithium-ion batteries are favourable largely due to the highly reductive nature of lithium. Taking the potential relative to the standard hydrogen electrode (SHE), the half-reaction of metallic lithium is approximately $-3.01V$ \cite{noauthor_battery_nodate}. The cells potential difference is the diference in potentals of the negative and posoitve solid state electrodes $V_t = \phi_s^+ (t) -  \phi_s^- (t) - IR_{cc}$ where $R_{cc}$ is the current collector or tabs ohmic resistance \cite{li_lithium-ion_2021}, having a large negeative reduction potential allows for a varierty of postive electordes, yeilding a high battery voltage. Coupled with Lithiums low molecular weight, a high energy density cell can be achived too.

All lithium-ion batteries follow the same basic principle of operation. They consist of a positive electrode, a negative electrode, an electrolyte, a separator, and current collectors. Unlike many battery chemistries where active materials undergo conversion reactions that change the electrode's chemical composition, lithium-ion electrodes typically store lithium through intercalation and de-intercalation — the lithium is inserted into or extracted from \cite{noauthor_battery_nodate}. The negative electrode is usually graphite, able to store up to one lithium atom per six carbon atoms. The positive electrode can vary greatly, the most common are often composed of transitional metal oxides such as LCO, LMO, NMC, NCA \cite{koech_lithium-ion_2024}. The electrochemical state of lithium in graphite is similar to that of metallic lithium, so the negative electrode potential lies close to -3.0 V vs. SHE. Positive-electrode lithium is in a lower electrochemical energy state, giving typical reaction potentials of roughly  0-1.3 V vs. SHE, overall there is a large potential difference created. The negative electrode is often the key limiting factor in degredation as discussed later. The posotive electrode materials vary greatly in material, but since this electrode has the biggest electrode potential, it has a largest impact on the overall cell voltage, and thus energy density of the cell, in addition since per volume it holds less lithium, electrode material with higer lithium capcity will be of greater affect that improving the negative electrodes capcity. The two reactions with lithium with the battery electrodes chosen for this project are shown below (during dicharge, the reactions proceede from left to right, for the charging process, it's right to left):

\begin{align}
  \ch{LiFePO4 &<=> Li+ + e- + FePO4} \label{eq:lithium} \\
  \ch{C6 + Li+ + e- &<=> LiC6} \label{eq:sodium}
\end{align}

The electrolyte is a medium that allows lithium ions to travel between the electrodes; however, it does not allow electrons to flow, which instead travel via the external circuit. The separator acts as a structural barrier to prevent the opposite electrodes from touching, allowing only ions to pass and thus preventing uncontrolled reactions that could cause major fire risks. During the discharging process, an external electrical path is provided between the collectors. The lithium stored within the negative electrode deintercalates, releasing lithium ions into the electrolyte, while the electrons leave the negative electrode, traveling across the external circuit to the positive electrode, where the lithium ions recombine with electrons and intercalate into the positive electrode structure. During charging, the applied voltage at the terminals is above the difference of the equilibrium electrode potentials. This applied overpotential drives lithium ions to deintercalate from the positive electrode, travel across the electrolyte, and intercalate into the negative electrode, where they combine with electrons. The speed at which the ions flow is much slower than the electron transfer, which contributes to the characteristic behavior of lithium batteries \cite{noauthor_battery_nodate}.

\begin{figure}[ht]
    \centering
    \includegraphics[width=0.4\textwidth]{../Images/LIStruct.png}
    \caption{Visial structure of Lithium Battery \cite{hasan_advancing_2025}}
    \label{fig:lipd}
\end{figure}

\subsection{Lithium based battery degredation modes}

J'O. Kane et al. \cite{jokane_lithium-ion_2022} summerises the key 3 modes of degredation caused within lithium ion batteries described below. Calendar aging is neglected in this discussion. 

\textbf{Lithium plating} - Under certain charging conditions, the lithium ions within the electrolyte will join with the electrons outside the negative electrode, producing pure lithium metal - which can grow formind dentrites, depicted in figure xx. The most understood causes which acclerate this is when the negative electrode potential falls below that of lithiums own electrode potential potnetial \cite{yang_minimum_2021,lain_understanding_2021}, becoming the most thermodynamically viable reaction. From electrochemical models, the potential at the elctrodes is the sum of their open circuit potentials and overpotential. The overpotential is largley due to the kenetic overpotential at the interface between the electroylte and electrode \cite{liu_insight_2021}, the Butler-Volmer equation shows as the current density $j$ increases, overpotential $\eta$, (figure \ref{fig:OP} shows the potentials changing duirg charge).Low temperatures and high SoC also increases Lithium plating, since it becomes harder for the lithium to incercalte within the carbon.
$$
j = a_s i_0 \left[ 
    \exp\!\left( \frac{\alpha_a F}{R T}\,\eta \right)
    - 
    \exp\!\left( -\frac{\alpha_c F}{R T}\,\eta \right)
\right]
$$



\begin{figure}[ht]
    \centering
    \includegraphics[width=0.3\textwidth]{../Images/Lithium_Plating.png}
    \caption{Lithium deposits shown in lighter grey on the graphite electrode \cite{attia_closed-loop_2020}}
    \label{fig:lip}
\end{figure}

\textbf{SEI Layer growth} - A layer known as solid electyroytle interphase is formed as soon as the electrolyte soulution comes into contact with the negative electrode causing salts like \ch{Li_{2}CO_{3}} to produce acids, followerd by futher reactions; this barrier acts to preven electrons further reducing and using up the more of the electrolyte, whilst allowing passage of the lithium ions to intercolate\cite{bouguern_critical_2024}. If this SEI breaks appart, new SEI will form, taking more lithium up, loosing material for charging (LLI). Having the battery at high and low SoC can cause the SEI layer to thicken also \cite{agubra_analysis_2014}. SEI formation also takes up electrons within its reaction, thus reducing battery capacity.

\textbf{Particle fracture} - The physical volume of the elctrodes can change during the interoclation and deintercalation, this is a degredation feature that cant be avoided in order to charge and discharge a battery, some electrodes exhibit more contraction than others, silicone is significantly greater than graphite\cite{noauthor_battery_nodate}. Concentraction gragrients, caused by high currents, within the elctrode can also cause internal stress. This can overtime cause the electrodes to break down, either resulting in more SEI grown, reducing more lithium; inability for areas of lithium intercalation, reducing charge capacity; and seperation from the binder causing either increased ohmic resistance or loss in capacity

In summary, high charging currents, extreme temperatures and extreme states of charge can accelerate degredaditon, thus to reduce this, charging and discharging at infetesimally small rates and at omptimal temperatures is best, however this neglects considdereations such as charge times, thus a balance between these constraints should always be taken into account and the charging stratagies used in this project attempts to do so. The impact on degredadtion can change between charging and discharging Mention diffusion differences between charge and discharge shown by J. Lain \textit{et al} \cite{lain_understanding_2021}.

\begin{figure}[ht]
    \centering
    \includegraphics[width=0.7\textwidth]{../Images/degred.jpg}
    \caption{Visial impact of common degredadition mechanisms \cite{birkl_degradation_2017}}
    \label{fig:deg}
\end{figure}

\subsection{Equivalent Circuit Model}

There are many ways a lithium battery can be modeled, depending on the accuracy requried, parameters avaoialable and computational power avaoialable. The two main categories are the equivalent circuit models and physics-based models \cite{noauthor_battery_nodate}. The physics-based models integrate conservation laws aswell as dynamical behaviour which leads to a set of PDE's with both scalar and gradient based boundary conditions. These require discritisation within the physical dimentions of the model to yeild a set of ODE's which can be solved for, PyBaMM is a popular framework for such modeling and its process is shwon in figure \ref{fig:pyb}. Popular physics based models in order of computational complexity \cite{noauthor_choosing_nodate} are the Doyle-Fuller-Newman model \cite{noauthor_doyle_2025}, Single-Particle Model with electolyte (SPMe) and SPM. The key problems of PBM is some require over 30 parameters to fully describe the properties and errors of such can accumulate over time, and depending on accuracy needed, the discitisation can yeild hundred of ODEs (cite battery desgin). 

\begin{figure}%
    \centering
    \subfloat[\centering PyBaMM solver pipeline]{\includegraphics[width=0.3\textwidth]{../Images/PyBaMM_pipe.jpg}}%
    \qquad
    \subfloat[\centering DNF Parameters]{\includegraphics[width=0.6\textwidth]{../Images/dfn_param.png}}%
    \caption{Parameters and solution steps for physics based modeling}%
    \label{fig:pyb}%
    \end{figure}

Equivalent circuit models use standard lumped electrical elements to match the behaviour to that of a real lithium battery dynamics, it is this type of model most common in BMS systems today. T.Kalogiannis \textit{et al} \cite{kalogiannis_comparative_2019} provides common methods for obtaining the parameter values used in ECM's. The most common version is shown in figure \ref{fig:ECM}, which is static in its dynamics. The key component is the $V_{oc}$ ideal voltage source, this value becomes a function, (when negleting temperature, degredaditon, charging-discharging hysterisis) of the batteries state of charge SoC, given by
$$ SoC(t) = z(t)/Q = (z(t_{0}) + \int_{t_{0}}^{t} \eta(\tau) i(\tau) \,d\tau)/Q $$ 
where $Q$ is the batteries nominal maximum capcity and $z$ is the current charge capacity. The $R_{0}$ and $R_{1}C_{1}$ branch are used to model the diffence in terminal voltage compared to that of the $V_{ocv}$ for a given instantanious SoC, and $\eta$ here is the colombic efficency, usually 0.99 for lithium batteries, since some current will be used in irreverisible chemical reactions within the cell\cite{noauthor_battery_nodate}. 

\begin{figure}[ht]
    \centering
    \includegraphics[width=0.5\textwidth]{../Images/ECM_FO.png}
    \caption{ECM model with 1st Order Polorizing dynamics}
    \label{fig:ECM}
\end{figure}

The common state space form of this system is given below

\begin{equation}
\begin{bmatrix}
\dot{v_{1}}{(t)} \\[4pt]
\dot{z}{(t)} \\[4pt]
\end{bmatrix}
=
\begin{bmatrix}
- \frac{1}{R_{1}C_{1}} & 0\\[4pt]
0 & 0 \\[4pt]
\end{bmatrix}
\begin{bmatrix}
{v_{1}{(t)}} \\[4pt]
{z{(t)}} \\[4pt]
\end{bmatrix}
+
\begin{bmatrix}
\frac{1}{C_{1}}\\[4pt]
\frac{1}{Q}\\[4pt]
\end{bmatrix}
i(t)
\end{equation} 

\begin{equation}
    V_{batt}(t) = V_{oc}(z) + i(t)R_{0} + v_{1}(t)
\end{equation}


W. Appia \textit{et al} \cite{appiah_unravelling_2024} provides the equation for the total overpotential within the battery derrived from the Fuller-Newman electrochemical mode, here, the total overpotential $\eta_{batt}$ is equal to the difference from the equlibrium battery potential $U_{batt}$ and terminal voltage potential $V_{batt}$, i.e $ V_{batt} = U_{batt} -\eta_{batt}$. This is analgous to the ECM model given above, whereby the overpotential (dynamical behaviour) is modeled by $i(t)R_{0} + v_{1}(t)$, where the equlibrium potential is $V_{oc}(z)$. 


\begin{equation}
\eta_{\text{batt}} =
\left[
\begin{aligned}
& 
\underbrace{
    \left( \Phi_{2,p}\big|_{x=L_p} - \Phi_{2,n}\big|_{x=0} \right)
}_{\text{Electrolyte overpotential } (\eta_2)}
\\[1.2em]
&+
\underbrace{
\left[
    \left( U_p(c_{1,p}^s)\big|_{x=L_p} - U_p(\bar c_p)\big|_{x=L_p} \right)
    -
    \left( U_n(c_{1,n}^s)\big|_{x=0} - U_n(\bar c_n)\big|_{x=0} \right)
\right]
}_{\text{Li concentration overpotential } (\eta_1^{c})}
\\[1.2em]
&+
\underbrace{
    \left( \eta^{ct}_p\big|_{x=L_p} - \eta^{ct}_n\big|_{x=0} \right)
}_{\text{Kinetic overpotential } (\eta^{ct})}
\;-\;
\underbrace{
    R_f I_{\text{app}}
}_{\text{Electrode ohmic overpotential } (\eta_1^{\Omega})}
\end{aligned}
\right]
\end{equation}

Whilst the voltage dynamics of lithium batteries do depend on temperature, this project keeps the ambient temperature to 30, thus realtime affects on the electrical dynamics are minimal. However moddeling of the temperature is vital since temperature does directly affect degredadion. L. Mattia \textit{et al.} \cite{mattia_lithium-ion_2025} explains the various thermal modelling aproaches and shows which contributions are key in heat production, with the main equation given below. Reverisible heat generation can not be directly infered by the basic ECM model shown in figure \ref{fig:ECM}. J. Xu \textit{et al.}, equated it being proportional to $\partial U / \partial T$ where $U$ is the OCV and $T$ is the temperature, this value does change over the SoC of the battery and paper et all calculates the curve for a battery of similar to the one used in this project, shown in figure \ref{fig:dudt}

\begin{equation}
    mCp\frac{d T}{dt} = \underbrace{I^{2}R_{0} + IV_{1}}_{irreversable} - \underbrace{IT\frac{\partial U}{\partial T}}_{reversable} - \underbrace{hAT}_{dissipation}
\end{equation}

\begin{figure}[ht]
    \centering
    \includegraphics[width=0.5\textwidth]{../Images/og_dudt.png}
    \caption{Charging segment extracted from the UoS dataset}
    \label{fig:dudt}
\end{figure}

A. Farman \textit{et al.}\cite{farmann_study_2017}, shows that the OCV of lithium batteries can change throughout degredaditon, Lithium phosphate batteries showed a 20mV difference in regions of the SoC after aging of 500 cycles at 1C of charge and discharge. Additionally, while there exits methods of extracting an accurate OCV curve, it can take hours to accuratley obtain to minimise overpotentials \cite{kalogiannis_comparative_2019}. This is where numerical methods could be utilised to obtain the OCV curve and the other ECM parameters by fitting against a grey-box model of the dynamics, ICLOCS2 \cite{noauthor_iclocs2_nodate}, will be used heavily throughout this project to see it's ability to parameterise the battery dynamics. A.A. Mohamed \textit{et al.} \cite{mohamed_advancements_2025} cite shows the many methods to obtain ECM parameters with methods ranging from analyitcal approaches based on various current inputs, to meta-heuristic optimization algorithms.  

\subsection{Charging methods}

There exists many techniques to charge batteries with a graphical summary of the common methods shwon in figure \ref{fig:Chargings}. Q. Lin \textit{et al.} \cite{chen_overpotential_2021} highliths the different methods described below. The most common method is known as CC-CV charge. This is where the bulk charge is done such that the applied overportnetial to the battery causes a constant current to flow, the voltage potential of the battery rises to a defined maximum point, the charging device then switches to a contant voltage charging mode, this fixed voltage is held until the charge current decays to a negligable or fixed amount - current still flows during this stage as there is still an overpotnetial between the batteries equlibrium state and the applied constant voltage. This is the simplest to implament without the need for an accurate SoC reading to risk overcharging (like would be with pure CC charging). 

\begin{figure}[ht]
    \centering
    \includegraphics[width=0.6\textwidth]{../../Matlab/Figures/CCCVProtocol.png}
    \caption{Charging segment extracted from the UoS dataset}
\end{figure}

CV is another method such that the voltage applied is the batterys maximum at full SoC, but since lithium batteries have low internal resistances, this can lead to a very high current draw at low SoC (where the equilibrium voltage is low), hence can cause accelerated degredadition, cite quan mentions 40\% capacity decrease after only 160 cycles. Boost charging provides a high current increase at low SoC, but only for breif periods and can significantly reduce degredadition rate compared to CV.

AC charging is another method which aims to minimise the frequency dependant impedance of the battery, effectivley allowing to charge the battery with a lower effective overpotentials, however as Q. Lin \textit{et al.} mentiones, there are disputes on the effectivness of this method; coupled with the fact EIS (electrochemical impedance spectroscopy) is rquried to find the frequency points of interest to target, this method wont be futher reseached within the project.

Multistage CC is a popular method in reseach as it allows for the customisation of the charging throughout the cycle, this means various models can be used to optimise for a specific goal. G. Tucker \textit{et al.} \cite{tucker_optimal_2023} for example, simulates various functions to minimise based of the ECM model provided in the section above, showing that the physics optimisation correlates closley to optimised degredadition obtained throuh purley data-driven methods. Electrochemical based models allows for a finer level of optimisation, such as ensuring the negative elecrode does not reach a potential to cause lithium plating, \cite{li_lithium-ion_2021} models an observer to estimate the potentials, which could be used within optimisation as one of the constraints, figure \ref{fig:OP} shows their predicited potential voltages. 

\begin{figure}[ht]
    \centering
    \includegraphics[width=0.6\textwidth]{../Images/Observer_op.png}
    \caption{Estimated potentials in response to charge current}
    \label{fig:OP}
\end{figure}

\begin{figure}[ht]
    \centering
    \includegraphics[width=0.5\textwidth]{../Images/common_charge_graph.jpg}
    \caption{Common charge techniques summarised}
    \label{fig:Chargings}
\end{figure}

The multistage CC charging method provides a good framework to trial different charging methods, its balance of ease of implamentation and adjustability means it is this method which will be used in the project to optimise for degredadition.

\subsection{Modelling degredation}

There are three main categorical methods to predict the degredation of lithium ion battereies: Purley data driven methods, physics-based models and a hyrbdid (phuysics informed). F. Wang \textit{et al.} \cite{wang_advanced_2025} highlitghts the advantages of the hybrid approach, stating issues and advantages with pure data-driven and pure physics model, they also present a means of categorising the common hybrid approaches. 

Papers focusing on the physics based moedeling usually focus on a subset of degredadion modes, such as SEI thickness growth cite Mixed Mode Growth Model for the Solid Electrolyte Interface. Most models start with the Doyle-Fuller-Newman model cite Dole Fuller to describe the transport of lithium within the battery (including potentials), then further sub-modules representing degredadition mechanisms are intergrated to yeild the required models. For example J. Okane \textit{et al.} \cite{jokane_lithium-ion_2022} adds SEI, Electrode cracking and loss of active material (LAM) as the the degredadion models. These approaches can model intericate parts within the battery, but require a lot of parameters about each battery under test, additionally, the discritised solutions are only ever as accurate as the models used.

Kuzhiyil \textit{et al.} \cite{kuzhiyil_lithium-ion_2025} by representing the electrochamiecal dynamics as $\frac{d\mathbf{x}}{dt} = f(\mathbf{x, z ,u}, t; \mathbf{\alpha})$ and couplining to degradation dynamics $\frac{d\mathbf{z}}{dt} = f(\mathbf{z, x ,u}, t; \mathbf{\theta})$ where $\mathbf{x} \in \mathbb{R}^{n_1}$ and $\mathbf{z} \in \mathbb{R}^{m_1}$ (once discritised), representing the battery electrochemical and degredadtion state variables respecitlvey by seperation of the speed of dynamics, 

Severson \textit{et al} \cite{severson_data-driven_2019}, provides one of the largest availaible dataset for battery charging. This paper takes uses a pruley data driven apprach in modeling, but uses enginererd features such as the change in discharge voltage curves between cycle 0 and cycle 100. Using a linear regression model, they predict the cycle-to-failure (defined when the capacity drops to 80\% of original), it shows upto a 9.2\% error. Attia \textit{et al.} \cite{attia_closed-loop_2020} takes the work of Severson \textit{et al.} to great use, using the model to predic how well differenct charge cycles last by cycling the batteries under random multistage CC cycles, predicting the degredadition and using a bayesian optmisation strategy\cite{noauthor_bayesian_nodate} to find the next appropriate charging protocol - in a closed loop feedback shown in figure \ref{fig:OptimB}

\begin{figure}[ht]
    \centering
    \includegraphics[width=0.6\textwidth]{../Images/atia.png}
    \caption{Closed loop workflow for finding optimal CC charging}
    \label{fig:OptimB}
\end{figure}

\section{Self Review}
Choosing this final year project has, and will continue to, be a big personal undertaking. I knew nothing about batteries other than the very basics when starting, I also knew very little within systems \& control other than simple LTI systems. Research has also been a struggle for me as many papers incorperate control methods or electrochemical background of which i dont have, a high amount of papers regarding the use of AI in modeling degredadtion are also behind paywalls - now more than ever on my 5th univerity year do I feel failing may well happen. I understand this document does not have indepth explinations of the use of AI within the modeling and may be vauge in the project steps, the project is very much in the sense adapting to what can be done. Work in general AI was done early in the project, I was able to learn the basics in pure nural-nets and produced a basic image classifer within pytorch but took a pause to work on other aspects of the project and univeristy modules. 

A large amount of time has been spent of narrowing down the focus of the project, as well as preparing the opmitised controll desgins for the lab testing. As of 01/12/2025, the first two batteries are undergoing characterisation tests and the softwear side is nearly done. Focus will now be spent on researching more about the machine learning and ways to optimise charging from the modeling of the battery. 


\begin{figure}[ht]
    \centering
    \includegraphics[width=1\textwidth]{../Images/Online Gantt 20251129.png}
    \caption{Gannt Chart of current project plan}
\end{figure}

\begin{figure}[htt]
    \centering
    \includegraphics[width=0.7\textwidth]{../Images/rr.png}
    \caption{Current risk register}
\end{figure}
\clearpage
\newpage
\nocite{*}
\textbf{Note: } The following references may have duplicates and unused references, this should hopefully be fixed for the final report
\bibliographystyle{../IEEEtran/IEEEtran}
\bibliography{My_Library}

\end{document}