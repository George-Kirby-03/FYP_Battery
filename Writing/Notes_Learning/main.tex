\documentclass[11pt]{article}
\usepackage[utf8]{inputenc}
\usepackage{graphicx}
\usepackage[margin=1in]{geometry}
\usepackage{amsmath}
\usepackage{hyperref}
\setlength{\parskip}{0.8em plus 0.2em minus 0.1em}
\setlength{\parindent}{0pt}
\usepackage{booktabs}
\usepackage{chemformula}
\renewcommand{\arraystretch}{1.2}



\linespread{1.5}
\title{
    \textbf{Notes FYP} \\[2em]
}
\author{George W. Kirby \\[2em] \textit{200328186} \\[2em] }
\date{\today}

\begin{document}
\maketitle
\vfill 
\begin{center}
    \textbf{Supervisor:} Dr. Ross Drummond 
\end{center}
\newpage

\section{Constraining the research}
Spent a fair amount of time learning about the basic machine learning techniques, then NN methods, then applying with pytorch. Managed to get a basic 
raw NN to work for fashion sets, then began looking at CNNS and LSTMS for learning data, in the hopes it could predict battery degredation over time. However,
given the sheer data needed, as well as a very large possible set of outputs and too many inputs to consider, it did not look feasable to continue down this route.
Atleast for a black box approach, to parameterise the current state of health, perhaps this could be used to live tune the current profile.

It was also found \cite{chinnam_fast-charging_2021} that differences of only 2\% can have large effects in the degredation states over time, 
meaning the ability for a NN to generalise well enough and capture these differences would be hard and more specifically, beyond the ability of the autor of this paper.

Instead of a \textit{black box battery} model, the goal is now to focus on the actuall optimal charging method themselfs, to reduce degredation. Specifically the
constant current stage of the charging cycle, as this is where most of the heat is generated, research shows this to be a large factor of degredatdion alongside instantatious
applied voltages. 

\begin{figure}[ht]
    \centering
    \includegraphics[width=0.5\textwidth]{../Images/nn_battery.png}
    \caption{Original end objective: Black Box Battery to allow for discorvery \& testing of optimal charging profiles}
\end{figure}

\section{Data Analysis on Dans Data}
Gave a good insight to the degredation patterns on an array of lithium batterys, data was analysed and plotted on jypter notebook.
Despite not complete draining ect, resting points, internal resistance and \textit{importantly} temperature were able to be extracted from the data too

\section{Lithium Battery Modelling}
Starting off, only knowing the basics of batteries, i.e the resistance increases over time, capacity drops ect. I'm continuing learning the various
battery models, behaviours ect.

\begin{itemize}
  \item For the most part, atleast within the context of the problem, the dynamics of the battery can be modelled with an equivalent circuit model (ECM). Subject to 
vary between cycles
\item Looking at dans data, parameters will be different between cells, as well as cycle degredation, but if the degredation can be modelled based off initial parameters, 
then an optimal charging method can be found for a given battery at a given time.
\item Degredation causes:
    \begin{itemize}
        \item SEI layer growth via pores $ \approx $ not really solvable, grows square root over time and cycle number
        \item Lithium plating \\
        Causes increased ageing and seftey risks, its the deposition of metallic lithoum on the anote surface, happens at high charging currents and low temperature.
        Since during charging, the lithium ions move , through the sei into the anode, if the ions cannot intercalate fast enough, they deposit and can become metallic lithium. 
        Especially ehrn chargis is forces, local overpotential can causes the lithium plating, can cause dentrites
        \textbf{this is one of the main constraints for the chargings profile}
        \item Active material loss (from parts mentiones above)
        \item SEI Brakeages \\
        Charging too \textbf{high} of a temperatures causes mechanical stress on the sei layer, causing it to crack and reform, consuming more lithium ions in the process. 
        Loose sei material can also float in the electrolyte, causing further issues.
        \item Electrolyte decomposition
    \end{itemize}
\end{itemize}

Superlinear battery degredadion known as "Knee" is where degredation drops rappidly over later cycles. 

Appears the multistage cc is advantaguous for keeping charge time down, yet reducing degredation by ensuring most of the current is
applied at lower states of charge, where the battery is less prone to lithium plating and high internal resistance heating.



\section{Current work and Results}
Looking at the paper on CLO, large question about the early predictor aoutcome,
mentions its a linear mechanism, how are they confimring what the characteristics are
after atleast the knee point? 

\section{Current plan}

 \begin{itemize}
        \item Look at existing charging methods, including the complex ones and continous ones (explain complexity and non generalisability). 
        \item Look at the different SOC estimation methods, since the cc high current section works good for 20-60 \% soc \cite{khan_multistage_2016}
        This could, and hopefully so, be a chance to use NN to predict soc quickly and something that can be implamented on hardware. Could also give chace to
        be compared against paings offline parameterisation solver
        \item If this is adaptive over the ageing, since R and C values change, need to look at maybe live cc tuning methods, maybe a form of MPC? , see the feasability
        of implamenting on actual hardware, explicit MPC could be a possability, but not sure yet how recomputing QP (or probably nonlinear) with changing dynamics is done
        \item Run the experiment against standard cc-cv methods, look at temp, internal resistance and capacity over time.
    \end{itemize}


     Baselinse batterys with fixed cc cv (need to look at the cc used)

Idea: Use ICLOCS2, paings model to extract features and the ECM parameters
Use this in a NN , possibly LSTM and NN to then allow for redicitoikn of furutre features

CC stages - follow roughly what Georges Paper ustilised to minimise the constrains, maybe
change the cost functions

Adapdtive, id like to be able to 

\section{Questions }

 \begin{itemize}
        \item Deciding on the constraints, besides the total charge volume, does the charge time need to be minimised also? Or keeping that constant 
        and purrley investigating the degredation effects compared to standard cc-cv method
        \item Enquire about dans temperature controll side, is the abient area controlled, can the temp be controlled?
        \item General guidance on the controll method, is this entire plan okay, any suggested reading? Some of the heavy matrices are a bit over my head. 
        (Happy with the idea of matracies transforming vectors, some basic forms of matrices w properties ect)
    \end{itemize}
\section{Porgress Log}
Entry 1: Terrible, a such fruitfull datasheet has made extracting parameters a breeze, infact, its taken such small amount of time, i have been free to complete all my
other modules, its so nice not to have to do anything

I definitley did not need to spend hours getting nowhere


Going to try to find a ocv cuve to help with the ICLOCS2 model, ittertive type method. I also know R1+R0.
Okay, so some okay progress been made, was really stuggling with nopthing giving to it, hard part is the graph isnt even complete!
So using this ocv curvce with a poly count of 12 seemed okay and a setting of 130
\begin{figure}[ht]
    \centering
    \includegraphics[width=1\textwidth]{../../Matlab/Figures/Fixed_ocv_curve.jpg}
    \caption{Fixed OCV curve from paper ... }
\end{figure}

The problem risided in the Resistance chosen, since ICLOCS is wanting to match the output, since the output during the middle of the SOC
can only be modified by the resistances (since the OCV curve is only generic, a large R was calculated to try fit). This meant that when 
the resistance was unbound, it made it look as if the parametes allowed a nice fit, but they were fitting an incorrect ocv curve. Thus,
one apprach was to atleast bound one resistance by being a function of the sumed resistance. This was possible since the datasheet gave a
discharge curve for various currents, thus, for the center soc section (where the graph wasnt cut off), the difference between the curves
should equal to the the resistiver drop across R1+R0. This allowed a decent estimate of R1+R0, and then R0 was bounded to be less than this value.


Between 0.1C and 3C dishcarge (at SOC 50\%), the voltage difference was 0.287, thus R1+R0 = 0.287/ (2.9C rate current) = 0.287/ (2.9*1.5A) = 0.066 Ohms.
Thus $R0 + R1 < 0.066 \Omega$. The initial drop in voltage would have allowed R1 to be estimated, however the graph is cut off too early to see this,
when taking the litteral values from the table, R1 is shown to be $0.09 \Omega$ which is not possible since R1 + R0 must be less than 0.066 Ohms.


Thus, the next approach was to parametrise only R0, and set R1 to be 0.066 - R0. Before attempting this, its expected a fair offset from the actual oputput
given the ocv curve discrepencies. Nevertheless, this was attempted. In order to implament this in iclocs, R1 was taken out of the parameterisation and refomrulated
as 0.066 - R0. It was discovered that then applying bounds on R0 caused larghe changes in the capaicty values, i.e a $10 \Omega$ change in R0 caused a near $1000F$ change in capaicty - 
its assumed this is due to the small available transients in the limited data, so a sensible bound of 0.01 to 0.04 Ohms was chosen, based on most ECM's showing similar R0 \& R1 values.
From this, atleast a Capaictance can be narrowred to a true value of 1500F +- 500F.

Now, the bounds sensiblys taken from the above results should hopefuuly allow the true ocv curve to be estimated, a 10th order polynomial was chosen to give enough flexability.


\begin{table}[ht]
\centering
\renewcommand{\arraystretch}{1.2}
\begin{tabular}{lccccc}
\toprule
\textbf{Case} & $\mathbf{Q\ (As)}$ & $\mathbf{C\ (F)}$ & $\mathbf{R_0\ (\Omega)}$ & $\mathbf{R_1\ (\Omega)}$ & $\mathbf{MSE}$ \\
\midrule
Fixed\_OCV\_0.1          & 5675 & \text{min bound} & 0.1   & 0.1   & 0.2 \\
Fixed\_OCV\_Unbound      & 5684 & 6085             & 0.24  & 0.133 & \text{d} \\
Fixed\_OCV\_R0\_Fix      & 4800 & 1024             & 0.012 & 0.008 & \text{d} \\
Fixed\_OCV\_R1\_Fix      & 5684 & 2841             & 0.007 & 0.003 & \text{d} \\
Fixed\_OCV\_Symetric     & 5685 & 1561             & 0.026 & 0.04 & \text{d} \\
\bottomrule
\end{tabular}
\vspace{3pt}
\caption{Parameter values for different estimation configurations.}
\label{tab:battery_params}
\end{table}

Very much sturggling, always seems to be fitting it to the ocv curtve, capaictance is just reallyt stuggling. Nex step, run it through dans currnet and simulate on ode45

On the thermal note, maybe see if it does affect a single cycle path for even more accuracy. From paper ... it shows that temperture dosent really affect the ocv cuve, mainly the internal resistance and maybe the ecm capaicotr.
Its a misconeption that the charge (capaicty) changes with temperature, its the ability to deliver that changes, i.e energy extraction, directly related to internal resistance increase and limilts on max current draw for sei for the 
cold temperatures too. (Cold temperatures are not looked at in this scope, mainly high temp effects during charge). Infact, this phenominan can be shoiwn on the discharge graph for the LiPo datasheet, higher current draw is causing a larger 
voltage drop, thus without risking damage to the cell, the 2.6V limit is reached sooner, showing less capacity effecitively drained.

\textbf{NOTE:} The ocv curve does indeded change during degredation and has been reported to significantly change esitmation models \cite{schmitt_capacity_2023}, will mean unfortunatley this cant be fixed (but we can atleast try)

So, next step is to just get the parameterised model to run better duriing current inputs via ode45 first.

\subsection{Thermal}

Since BIOT number is very small, we can assume the temp within the cell is uniform, thus a lumped model can be used \cite{wang_critical_2016}. Heat itself if formed by the following:
\begin{equation}
    Q_{gen} = I^2 R + I \left( \frac{\partial U}{\partial T} \right)_{soc}
\end{equation}
\begin{equation}
    Q_{gen} = I(Uoc - V) - I  \left(T \frac{dUoc}{dT}\right)
\end{equation}
\begin{itemize}
    \item Joule Heating: $I^2 R$ : Resistive heating from internal resistance
    \item Entropc (reversable) heating: $I \left( \frac{\partial U}{\partial T} \right)_{soc}$ : Caused by the entropy change during the electrochemical reactions,
    \item 
\end{itemize}

Yya, got those styuff done, will preobs auytmate on all dans to get cp, l values

Anyways, for the cyurrent sims, before the mpc which im still stuffed for, i can discriteise mysen and ill run it thru a 3 part soc charge too
so im gonna make a (profile maker) function



Points for next meeting:

Concern on the simulation: Capacity valueus and OCV curve accuracy (shouw variations with pinning SOC(0) voltage, forced capacity, and free rein)
Show current sim results
Show stages of charging simed ode45
Discriticisation 
(Explain working on mpc)
\begin{equation}
    \frac{\partial U}{\partial T} \approx f(SoC) = f(z)
\end{equation}
\begin{equation}
    mCp\frac{d T}{dt} = I^{2}R_{ecm} - ITf(z) - hAT
\end{equation}
\begin{equation}
f(z,z_{0}) \sim c_{0} + zc_{1} + z^{2}c_{2} + z^{3}c_{3} + z^{4}c_{4}
\end{equation}
\begin{equation}
    mCp\frac{d T}{dt} = I^{2}R_{0} + IV_{1} - IT(c_{0} + zc_{1} + z^{2}c_{2} + z^{3}c_{3} + z^{4}c_{4}) - hAT
\end{equation}
\begin{equation}
\begin{bmatrix}
\dot{T} \\[4pt]
\dot{z} \\[4pt]
\dot{V_{1}}
\end{bmatrix}
=
\begin{bmatrix}
I^{2}R_{0} + IV_{1} - IT\frac{1}{mCp}(c_{0} + zc_{1} + z^{2}c_{2} + z^{3}c_{3} + z^{4}c_{4}) - hAT\frac{1}{mCp} \\[4pt]
I/Q \\[4pt]
I/C- V_{1}/R_{1}C
\end{bmatrix}
\end{equation} 





B01Charac R0 R1 points : 84586 - 84989, 120362 - 120713, 251964 - 252328, 240052 - 240374


B02Charac R0 R1 points : 85540 - 85940, 109372 - 109834, 205557 - 205973, 241438 - 241816
35134



\subsection{Normalising attia current profile}

The battery chosen is different to the one from attia, and has a higher resistance and lower charging capaicity, therefore, the limits and most optimal profile for the attia has to be normalised to the battery we have. This isnt trivial, since there are many variables that are set in the attia framework and as many as possible should be approprately matched. The framework from attia is as follows the charging is broken into 5 segments, with the lower 4 been CC style, and the 4th as a CC-CV section. Each section is seperated by the SoC of which the charge value occupiys, each section takes up 20\% SoC, therfore the CC section values can be denoted as \textbf{CC1}, \textbf{CC2}, \textbf{CC3} and  \textbf{CC4}. 

\textbf{CC1}, \textbf{CC2}and \textbf{CC3} are variables which can be directly optimised and controlled, subject to their respective upper bound which is limited in the attia case to not reach the batteries upper OCV voltage during the charging stages (in attia case this is 3.6V), and their is a constant lower bound for these three sections also. \textbf{CC4} has the same upper bound constraint definition, yet its defined value depends only on the given values for \textbf{CC1}, \textbf{CC2} and \textbf{CC3}. This allows the charge duration from SoC 0-80\% to be fixed whilst allowing the charging current during the SoC ranges to be modified. The equality that must be held (before variable constraints), given in attia, is given as ...
$$ t_{0-80\%} = 0.2\left( \frac{1}{CC1} + \frac{1}{CC2} + \frac{1}{CC3} + \frac{1}{CC4}\right)$$

To provide inequalities to help decide values, two cases were considered to help to help reduce descisions, since all $\mathbf{CC}_i$ sections have there maximum value limited ($\mathbf{CC}_{i,\max}$) physically, these can be calculated to help define the minimum bounds and $t_{0-80\%}$. 

The first case considered is when $\mathbf{CC}_i$, $i = 1,2,3$, are at their maximum, since all currents are posotive values, the equality given above must mean that $\mathbf{CC}_4$ is at its minimum. This inequality narrows descision variables to only 2, with the equality given as $$\mathbf{CC}_{4,\min}\times\left(t_{0-80\%} - 0.2\sum_{i=1}^{3} \frac{1}{CC_i,max}\right) = 0.2$$. The maximum permissable current values can be worked out by finding for each stage along a charging profile, the current that can be applied to bring the OCV voltage to the upper bound (difference is just $IR_{0+1}$), using the OCV curve obtained gives the graph below, the values were obtained in C units: $\mathbf{CC}_{1,\max} = 3.4$, $\mathbf{CC}_{2,\max} = 2.9$, $\mathbf{CC}_{3,\max} = 2.8$ and $\mathbf{CC}_{4,\max} = 2.5$. Thus the equation has only two variables and can be written as  $\mathbf{CC}_{4,\min} = \frac{0.2}{t_{0-80\%} - 0.279}$

In the second case, supose there is a current profile which is lowest for all $\mathbf{CC}_i$, $i = 1,2,3$ (which as mentioned is the same for those segments, thus is denoted as $\mathbf{CC}_{1:3,min}$), again due to the posotive nature of the currents, must mean $\mathbf{CC}_{4}$ is at its maximum. Since as before $\mathbf{CC}_{4,max}$ is known, a similar equation as before is calculated as $\mathbf{CC}_{1:3,min} = \frac{0.2\times3}{t_{0-80\%} - \frac{0.2}{\mathbf{CC}_{4,max}}} = \frac{0.6}{t_{0-80\%} - 0.08}$. These two equations can now be plotted to help decide values based on $t_{0-80\%}$.

\begin{figure}[ht]
    \centering
    \includegraphics[width=1\textwidth]{../Images/msad.png}
    \caption{Fixed OCV curve from paper ... }
\end{figure}

This shows that if the charging duration is increase, it allows for a greater range of the available currents at different SoC ranges which is advantaguous to find the optimal protocols, however this does increase charge time and lab time is of priority given the FYP time. 

The above work shows how the minimum ranges for the segments can be derived, clearley, however these upper and lower bounds and times do not equate to Attias values - they only provide values in the same matter Attia derived theirs, and time still needs finding. A descision is now made on how to scale the charge protocols provided by attia, if the CC upper segments are scaled by the ratio of $\mathbf{CC}_{1,attia,max} / \mathbf{CC}_{1,RS,max}$, since both batteries are of similar chemistry, the ratio of upper CC limits nearly coicide with the RS battery. A table is shown below to highlights this. Only $\mathbf{CC}_{2}$ is slighly over the limit. 

\begin{table}[ht]
\centering
\renewcommand{\arraystretch}{1.2}
\begin{tabular}{lcccc}
\toprule
\textbf{CC\_max\ values} & $\mathbf{CC}_{1,\max}$ & $\mathbf{CC}_{2,\max}$ & $\mathbf{CC}_{3,\max}$ & $\mathbf{CC}_{4,\max}$ \\
\midrule
$CC_{rs max} \text{ Actual}$ & 3.4 & 2.9 & 2.8 & 2.5 \\
$CC_{1,\text{attia},\max}/CC_{1,\text{rs},\max} \text{ Norm}$ & 3.4 & 2.97 & 2.38 & 2.04 \\
\bottomrule
\end{tabular}
\vspace{3pt}
\caption{Parameter values for different estimation configurations.}
\label{tab:Cs}
\end{table}

The duration of $t_{0-80\%}$ in Attia is 10 minuets, looking at figure x, this clearley can not be achived, if, the scaling used in table Cs row 2 is used as the upper limits are slightly reduced futher, yeilding ever worse achivability. However by scaling the CC\_max that way, the CC values obtained by Attia can be siply each devided by the same scaling factor (2.35 in this case). This then forces the $t_{0-80\%}$ to be 23.5 minuets. For example, the best CC segments from attia are 5.2C-5.2C-4.8C-4.16C, since the scaling of CC\_max is 2.35, the CC segments for out battery can be 2.2C-2.2C-2C-1.3C at takes 23.5 minuets. This is acceptable, however, 23.5 minuets limits the lower bounds on the CC segments, thus reducing the possible combinations to try optimising. To show this, the dashed lines are added to figure x which use the normalised CC\_max values in table C, and at $t_{0-80\%}$ = 0.39 would give the lower limits of $\mathbf{CC}_{1;3min}$ (2.43C) and $\mathbf{CC}_{4min}$ (2.06C). This is a problem straight away, this gives 0 variance to $\mathbf{CC}_{4}$. This poses a problem in translating the Attia framework, the cycles could be futher reduced in current by a scale factor greater than that which can scale the CC\_max, but these optimum values were obtained with these limits in place. It is therefore a balance between matching the trend of the attia protocols, and current ranges to allow more variation. 

An idea is to subjectivley choose $t_{0-80\%}$ from figure x, and for the attia protocols, scale the $\mathbf{CC}_{1;3}$ segments by a factor which causes $\mathbf{CC}_{1;4}$ to be as close as possible in ratio to that of $CC_{1,\text{attia},\max}/CC_{1,\text{rs},\max}$, the $\mathbf{CC}_{4}$ value is to be calculated inline with methods before and attia - constrained to meet 80\% SoC, in $t_{0-80\%}$ time. This can very simply represented at $t_{0-80\%} = 0.2x() $

\nocite{*}
\newpage
\bibliographystyle{../IEEEtran/IEEEtran}
\bibliography{../Interim_Report/My_Library}

\end{document}