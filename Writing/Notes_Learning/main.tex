\documentclass[12pt]{article}
\usepackage[utf8]{inputenc}
\usepackage{graphicx}
\usepackage[margin=1in]{geometry}
\usepackage{amsmath}
\usepackage{hyperref}
\setlength{\parskip}{0.8em plus 0.2em minus 0.1em}
\setlength{\parindent}{0pt}

\linespread{1.5}
\title{
    \textbf{Notes FYP} \\[2em]
}
\author{George W. Kirby \\[2em] \textit{200328186} \\[2em] }
\date{\today}

\begin{document}
\maketitle
\vfill 
\begin{center}
    \textbf{Supervisor:} Dr. Ross Drummond 
\end{center}
\newpage

\section{Constraining the research}
Spent a fair amount of time learning about the basic machine learning techniques, then NN methods, then applying with pytorch. Managed to get a basic 
raw NN to work for fashion sets, then began looking at CNNS and LSTMS for learning data, in the hopes it could predict battery degredation over time. However,
given the sheer data needed, as well as a very large possible set of outputs and too many inputs to consider, it did not look feasable to continue down this route.
Atleast for a black box approach, to parameterise the current state of health, perhaps this could be used to live tune the current profile.

It was also found \cite{chinnam_fast-charging_2021} that differences of only 2\% can have large effects in the degredation states over time, 
meaning the ability for a NN to generalise well enough and capture these differences would be hard and more specifically, beyond the ability of the autor of this paper.

Instead of a \textit{black box battery} model, the goal is now to focus on the actuall optimal charging method themselfs, to reduce degredation. Specifically the
constant current stage of the charging cycle, as this is where most of the heat is generated, research shows this to be a large factor of degredatdion alongside instantatious
applied voltages. 

\begin{figure}[h]
    \centering
    \includegraphics[width=0.5\textwidth]{../Images/nn_battery.png}
    \caption{Original end objective: Black Box Battery to allow for discorvery \& testing of optimal charging profiles}
\end{figure}

\section{Data Analysis on Dans Data}
Gave a good insight to the degredation patterns on an array of lithium batterys, data was analysed and plotted on jypter notebook.
Despite not complete draining ect, resting points, internal resistance and \textit{importantly} temperature were able to be extracted from the data too

\section{Lithium Battery Modelling}
Starting off, only knowing the basics of batteries, i.e the resistance increases over time, capacity drops ect. I'm continuing learning the various
battery models, behaviours ect.

\begin{itemize}
  \item For the most part, atleast within the context of the problem, the dynamics of the battery can be modelled with an equivalent circuit model (ECM). Subject to 
vary between cycles
\item Looking at dans data, parameters will be different between cells, as well as cycle degredation, but if the degredation can be modelled based off initial parameters, 
then an optimal charging method can be found for a given battery at a given time.
\item Degredation causes:
    \begin{itemize}
        \item SEI layer growth via pores $ \approx $ not really solvable, grows square root over time and cycle number
        \item Lithium plating \\
        Causes increased ageing and seftey risks, its the deposition of metallic lithoum on the anote surface, happens at high charging currents and low temperature.
        Since during charging, the lithium ions move , through the sei into the anode, if the ions cannot intercalate fast enough, they deposit and can become metallic lithium. 
        Especially ehrn chargis is forces, local overpotential can causes the lithium plating, can cause dentrites
        \textbf{this is one of the main constraints for the chargings profile}
        \item Active material loss (from parts mentiones above)
        \item SEI Brakeages \\
        Charging too \textbf{high} of a temperatures causes mechanical stress on the sei layer, causing it to crack and reform, consuming more lithium ions in the process. 
        Loose sei material can also float in the electrolyte, causing further issues.
        \item Electrolyte decomposition
    \end{itemize}
\end{itemize}

Superlinear battery degredadion known as "Knee" is where degredation drops rappidly over later cycles. 

Appears the multistage cc is advantaguous for keeping charge time down, yet reducing degredation by ensuring most of the current is
applied at lower states of charge, where the battery is less prone to lithium plating and high internal resistance heating.



\section{Current work and Results}
Looking at the paper on CLO, large question about the early predictor aoutcome,
mentions its a linear mechanism, how are they confimring what the characteristics are
after atleast the knee point? 

\section{Current plan}

 \begin{itemize}
        \item Look at existing charging methods, including the complex ones and continous ones (explain complexity and non generalisability). 
        \item Look at the different SOC estimation methods, since the cc high current section works good for 20-60 \% soc \cite{khan_multistage_2016}
        This could, and hopefully so, be a chance to use NN to predict soc quickly and something that can be implamented on hardware. Could also give chace to
        be compared against paings offline parameterisation solver
        \item If this is adaptive over the ageing, since R and C values change, need to look at maybe live cc tuning methods, maybe a form of MPC? , see the feasability
        of implamenting on actual hardware, explicit MPC could be a possability, but not sure yet how recomputing QP (or probably nonlinear) with changing dynamics is done
        \item Run the experiment against standard cc-cv methods, look at temp, internal resistance and capacity over time.
    \end{itemize}


     Baselinse batterys with fixed cc cv (need to look at the cc used)

Idea: Use ICLOCS2, paings model to extract features and the ECM parameters
Use this in a NN , possibly LSTM and NN to then allow for redicitoikn of furutre features

CC stages - follow roughly what Georges Paper ustilised to minimise the constrains, maybe
change the cost functions

Adapdtive, id like to be able to 

\section{Questions }

 \begin{itemize}
        \item Deciding on the constraints, besides the total charge volume, does the charge time need to be minimised also? Or keeping that constant 
        and purrley investigating the degredation effects compared to standard cc-cv method
        \item Enquire about dans temperature controll side, is the abient area controlled, can the temp be controlled?
        \item General guidance on the controll method, is this entire plan okay, any suggested reading? Some of the heavy matrices are a bit over my head. 
        (Happy with the idea of matracies transforming vectors, some basic forms of matrices w properties ect)
    \end{itemize}
\section{Porgress Log}
Entry 1: Terrible, a such fruitfull datasheet has made extracting parameters a breeze, infact, its taken such small amount of time, i have been free to complete all my
other modules, its so nice not to have to do anything

I definitley did not need to spend hours getting nowhere
\nocite{*}
\bibliographystyle{../IEEEtran/IEEEtran}
\bibliography{../Interim_Report/My_Library}

\end{document}