\documentclass{article}
\usepackage[utf8]{inputenc}
\usepackage{graphicx}
\usepackage[margin=1in]{geometry}
\usepackage{amsmath}
\usepackage{hyperref}


\title{
    \textbf{Notes FYP} \\[2em]
}
\author{George W. Kirby \\[2em] \textit{200328186} \\[2em] }
\date{\today}

\begin{document}
\maketitle
\vfill 
\begin{center}
    \textbf{Supervisor:} Dr. Ross Drummond 
\end{center}
\newpage

\section{Constraining the research}
Instead of a \textit{black box battery} model, the goal is now to focus on the actuall optimal charging method themselfs, to reduce degredation. Specifically the
constant current stage of the charging cycle, as this is where most of the heat is generated, research shows this to be a large factor of degredatdion alongside instantatious
applied voltages. 

\section{Data Analysis on Dans Data}
Gave a good insight to the degredation patterns on an array of lithium batterys, data was analysed and plotted on jypter notebook.
Despite not complete draining ect, resting points, internal resistance and \textit{importantly} temperature were able to be extracted from the data too

\section{Lithium Battery Modelling}
Starting off, only knowing the basics of batteries, i.e the resistance increases over time, capacity drops ect. I'm continuing learning the various
battery models, behaviours ect.

\begin{itemize}
  \item For the most part, atleast within the context of the problem, the dynamics of the battery can be modelled with an equivalent circuit model (ECM). Subject to 
vary between cycles
\item Looking at dans data, parameters will be different between cells, as well as cycle degredation, but if the degredation can be modelled based off initial parameters, 
then an optimal charging method can be found for a given battery at a given time.
\item Degredation causes:
    \begin{itemize}
        \item SEI layer growth via pores $ \approx $ not really solvable, grows square root over time and cycle number
        \item Lithium plating \\
        Causes increased ageing and seftey risks, its the deposition of metallic lithoum on the anote surface, happens at high charging currents and low temperature.
        Since during charging, the lithium ions move , through the sei into the anode, if the ions cannot intercalate fast enough, they deposit and can become metallic lithium. 
        Especially ehrn chargis is forces, local overpotential can causes the lithium plating, can cause dentrites
        \textbf{this is one of the main constraints for the chargings profile}
        \item Active material loss (from parts mentiones above)
        \item SEI Brakeages \\
        Charging too \textbf{high} of a temperatures causes mechanical stress on the sei layer, causing it to crack and reform, consuming more lithium ions in the process. 
        Loose sei material can also float in the electrolyte, causing further issues.
        \item Electrolyte decomposition
    \end{itemize}
\item 
\end{itemize}



Superlinear battery degredadion known as "Knee" is where degredation drops rappidly over later cycles. 
\section{Machine learning techniques}
Spent a fair amount of time learning about the basic machine learning techniques, then NN methods, then applying with pytorch. Managed to get a basic 
raw NN to work for fashion sets, then began looking at CNNS and LSTMS for learning data, in the hopes it could predict battery degredation over time. However,
given the sheer data needed, as well as a very large possible set of outputs and too many inputs to consider, it did not look feasable to continue down this route.
Atleast for a black box approach, to parameterise the current state of health, perhaps this could be used to live tune the current profile.



\section{Current work and Results}


\section{Current plan}



\section{Self Assessment}


\newpage
\nocite{*}
\bibliographystyle{../IEEEtran/IEEEtran}
\bibliography{../Interim_Report/My_Library}

\end{document}