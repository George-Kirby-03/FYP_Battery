\documentclass[11pt]{article}
\usepackage[utf8]{inputenc}
\usepackage{graphicx}
\usepackage[margin=1in]{geometry}
\usepackage{amsmath}
\usepackage{hyperref}
\setlength{\parskip}{0.8em plus 0.2em minus 0.1em}
\setlength{\parindent}{0pt}
\usepackage{booktabs}
\usepackage{chemformula}
\renewcommand{\arraystretch}{1.2}



\linespread{1.5}
\title{
    \textbf{Notes FYP} \\[2em]
}
\author{George W. Kirby \\[2em] \textit{200328186} \\[2em] }
\date{\today}

\begin{document}
\maketitle
\vfill 
\begin{center}
    \textbf{Supervisor:} Dr. Ross Drummond 
\end{center}
\newpage



\subsection{Background}

The goal is to find an optimal CC multistage charging protocol for a battery; inline with the Attia paper the duration of the charge from $SOC_{0\%}$ to $SOC_{80\%}$ is fixed, and the CC segments are from 0-20, 20-40, 40-60, 60-80 and 80-100, with the 80-100 been detimined based off the previous segments in order to reach the fixed charge time. 

Currently the model used for the battery is from the ecm 1st order polorising model, defined by $Cap, C, R0, R1$ and $OCV(soc)$ (ocv curve in high order smooth polynomial form) - which at this point are already obtained. A 1-d Lumped element thermal model is also used for temperature predicitons. Additionally, for our RS model, the duration of the charge from $SOC_{0\%}$ to $SOC_{80\%}$ has also been calculated to meet the limits of the battery. Thus there is enough information to run the optmisations.


\begin{table}[ht]
\centering
\renewcommand{\arraystretch}{1.2}
\begin{tabular}{lcccc}
\toprule
\textbf{CC\_max\ values} & $\mathbf{CC}_{1,\max}$ & $\mathbf{CC}_{2,\max}$ & $\mathbf{CC}_{3,\max}$ & $\mathbf{CC}_{4,\max}$ \\
\midrule
$CC_{rs max} \text{ Actual}$ & 3.4 & 2.9 & 2.8 & 2.5 \\
$CC_{1,\text{attia},\max}/CC_{1,\text{rs},\max} \text{ Norm}$ & 3.4 & 2.97 & 2.38 & 2.04 \\
\bottomrule
\end{tabular}
\vspace{3pt}
\caption{Parameter values for different estimation configurations.}
\label{tab:Cs}
\end{table}

The duration of $t_{0-80\%}$ in Attia is 10 minuets, looking at figure x, this clearley can not be achived, if, the scaling used in table Cs row 2 is used as the upper limits are slightly reduced futher, yeilding ever worse achivability. However by scaling the CC\_max that way, the CC values obtained by Attia can be siply each devided by the same scaling factor (2.35 in this case). This then forces the $t_{0-80\%}$ to be 23.5 minuets. For example, the best CC segments from attia are 5.2C-5.2C-4.8C-4.16C, since the scaling of CC\_max is 2.35, the CC segments for out battery can be 2.2C-2.2C-2C-1.3C at takes 23.5 minuets. This is acceptable, however, 23.5 minuets limits the lower bounds on the CC segments, thus reducing the possible combinations to try optimising. To show this, the dashed lines are added to figure x which use the normalised CC\_max values in table C, and at $t_{0-80\%}$ = 0.39 would give the lower limits of $\mathbf{CC}_{1;3min}$ (2.43C) and $\mathbf{CC}_{4min}$ (2.06C). This is a problem straight away, this gives 0 variance to $\mathbf{CC}_{4}$. This poses a problem in translating the Attia framework, the cycles could be futher reduced in current by a scale factor greater than that which can scale the CC\_max, but these optimum values were obtained with these limits in place. It is therefore a balance between matching the trend of the attia protocols, and current ranges to allow more variation. 

An idea is to subjectivley choose $t_{0-80\%}$ from figure x, and for the attia protocols, scale the $\mathbf{CC}_{1;3}$ segments by a factor which causes $\mathbf{CC}_{1;4}$ to be as close as possible in ratio to that of $CC_{1,\text{attia},\max}/CC_{1,\text{rs},\max}$, the $\mathbf{CC}_{4}$ value is to be calculated inline with methods before and attia - constrained to meet 80\% SoC, in $t_{0-80\%}$ time. This can very simply represented at $t_{0-80\%} = 0.2/x_{scale}\times(\left( \sum_{i=1}^{3}\mathbf{CC}_{i}^{-1} \right) + \mathbf{CC}_{4}^{-1})$, which can be written in a graphical form as $$\mathbf{CC}_{4} = 0.2 / (x_{scale}t_{0-80\%} - 0.2\left( \sum_{i=1}^{3}\mathbf{CC}_{i}^{-1} \right))$$. The right hand side can be plotted and yeilds the following (Y value is $t_{0-80\%}$), the optimised choice is such that the scalar value in the feild is as close to $\mathbf{CC}_{4}$ of the chosen Attia profile (the most optimal version they found for example), so $\mathbf{CC}_{4} = 4.16$ in the chosen protocol, alongside the respective $\mathbf{CC}_{1;3}$ values, the point which is scaled given in the previous pharagraph is shown in red, but as mentioned it limits the potential currents for more optimisations and is too short of a charge time. The point decided keeps the scaling of  $\mathbf{CC}_{4}$ almost identical to that of the other segments, thus keeping the pattern correct, whilst choosing a scaling factor as close as possible to that in the previous pharagraph without been to short of charging. A balance was thus decided, with a $t_{0-80\%}$ of 0.55 chosen, this, refering back to the previous graph still allows for a large range of current choices with a low enough C\_min set of values. Thus, for the Attia optimal protocol, the CC values are \textbf{1.73-1.73-1.6-1.39}.
%
%\begin{figure}[ht]
 %   \centering
%    \includegraphics[width=1\textwidth]{../../Matlab/Learning/Dans_Data_Extraction/Lab_cycles/Scaling found.png}
%    \caption{C4 Time Scaling balance}
%\end{figure}

Lastly the final CV section is to be chosen, this could have been scaled with more reasoning, but for now a value of 0.5C (half of the Attia value was chosen for now to balance total charge duration and the scaling of the previous segments), Attia value was fixed to 1C.
%

\nocite{*}
\newpage


\end{document}